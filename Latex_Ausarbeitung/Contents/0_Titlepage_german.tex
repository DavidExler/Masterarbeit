% !TeX root = ../Thesis.tex
\pagenumbering{Roman}

\pagenumbering{Roman}
\begin{titlepage}
\ThisCenterWallPaper{1}{Logo/title-background.pdf}
\begin{center}
  \includegraphics[width=0.4\textwidth]{Logo/KITlogo_4c_deutsch.eps}\\
  \vspace{6pt}
  \large{Fakultät für Maschinenbau}\\
  \vspace{6pt}
  \large{Institut für Automation und angewandte Informatik}\\
  \vspace{18pt}
  \Large{\textbf{Optimierung von Deep-Learning-Methoden}\\ \textbf{zur Auswertung biologischer 3D-Bildstapel}}\\
  \par\vfill
  \begin{onehalfspacing}
  \normalsize{Masterarbeit in Elektrotechnik und Informationstechnik\\
  eingereicht von\\
  David Exler\\
  Martikelnummer: 2499363\\
  \vspace{12pt}
  \begin{tabular}{ll}
  Erstprüfer:   & Prof. Dr. Gerardo Hernandez-Sosa\\
  Zweitprüfer: & Prof. Dr. Markus Reischl\\
\end{tabular}\\
  \vspace{12pt}
  2025\\}
\end{onehalfspacing}
\end{center}
\end{titlepage}

\cleardoublepage

\thispagestyle{empty}
\begin{quote}
\small
\textbf{Abstract}\quad 
Powerful deep learning methods are often necessary for the evaluation of biological 3D image data, as manual evaluation is complex and time-consuming.
In many areas of biological research, the evaluation of 3D image stacks is essential, but customized and optimized deep learning methods are not available. \newline
This paper deals with the automated optimization of 3D classification and 3D instance segmentation methods.
A new metric for instance segmentation models for biological image data is introduced, which evaluates models in terms of their ability to extract interpretable features such as cell nucleus count or local cell nucleus densities.
In addition, new 3D classification methods are introduced that can adapt a classification model to different features of a dataset from biological 3D image stacks. 
These methods include preprocessing methods, pretraining methods, and classification heads.
In addition, a new application for assisted class annotation of 3D cell data is presented. \newline
The entire methodology is implemented in an application, the 3D cell data pipeline.
The application enables biology experts without programming knowledge to optimize the aforementioned deep learning methods for biological 3D image stacks.
As a case study, the 3D cell data pipeline is applied to a dataset of myotube cultures with fluorescent markers for cell nuclei, muscle strings, and Schwann cells.
The results of these experiments show that the application efficiently finds optimal deep learning methods and significantly improves the accuracy of the evaluation of biological 3D image stacks.
\clearpage
\end{quote}


{\selectlanguage{ngerman}
\begin{quote}
\small
\textbf{Zusammenfassung}\quad 
Zur Auswertung von biologischen 3D-Bilddaten sind häufig leistungsstarke Deep-Learning-Methoden notwendig, da die manuelle Auswertung komplex und zeitintensiv ist.
In vielen Bereichen der biologischen Forschung ist die Auswertung von 3D-Bildstapeln unerlässlich, jedoch sind angepasste und optimierte Deep-Learning-Methoden nicht verfügbar. \newline
Die vorliegende Arbeit behandelt die automatisierte Optimierung von \linebreak 3D-Klassifikations- und 3D-Instanzsegmentierungsmethoden.
Es wird eine neue Metrik für Instanzsegmentierungsmodelle für biologische Bilddaten eingeführt, die Modelle hinsichtlich ihrer Eignung, interpretierbare Merkmale wie die Zellkernanzahl oder lokale Zellkerndichten zu extrahieren, bewertet.
Außerdem werden neue 3D-Klassifikationsmethoden eingeführt, die ein Klassifikationsmodell an verschiedene Merkmale eines Datensatzes aus biologischen 3D-Bildstapeln anpassen können. 
Diese Methoden umfassen Vorverarbeitungsmethoden, Vortrainingsmethoden und Klassifikations-Köpfe.
Zusätzlich wird eine neue Anwendung zur unterstützten Klassen-Annotation von 3D-Zelldaten vorgestellt.\newline
Die gesamte Methodik ist in einer Anwendung, der 3D-Zelldaten-Pipeline, implementiert.
Durch die Anwendung wird Biologie-Expert*Innen ohne Programmierkenntnisse die Optimierung der genannten Deep-Learning-Methoden für biologische 3D-Bildstapel ermöglicht.
Als Fallstudie wird die 3D-Zelldaten-Pipeline auf einen Datensatz von Myotubenkulturen mit Fluoreszenzmarkern für Zellkerne, Muskelfasern und Schwannzellen angewandt.
Die Ergebnisse dieser Experimente legen dar, dass die Anwendung effizient optimale Deep-Learning-Methoden findet und die  Genauigkeit der Auswertung biologischer 3D-Bildstapel signifikant verbessert.

\clearpage
\end{quote}
}

\selectlanguage{ngerman}

\cleardoublepage


\cleardoublepage
{
\hypersetup{linkcolor=black}
\tableofcontents
}
\pagestyle{scrheadings}
\cleardoublepage