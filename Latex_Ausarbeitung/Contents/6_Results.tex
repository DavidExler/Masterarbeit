% !TeX root = ../Thesis.tex

\chapter{Results} \label{ch:results}
%In this chapter, the results are shown and discussed. Use an appropriate presentation of the results, e.g., tables, diagrams, plots. Be aware of statements that are too general.


\section{Überblick}

\section{Segmentierung}
Für die Wahl eines Segmentierungsnetzes wird in Kapitel \ref{ch:NewMethods} das Bewertungskriterien \ac{ipq} eingeführt.
Außerdem wird in Kapitel \ref{ch:NewMethods} der annotierte S\_BIAD1518 Datensatz vorgestellt.
Die \ac{ipq} wird auf dem Datensatz mit den Masken von drei vortainierten Segmentierungsnetzen und, zur Validierung, mit der \ac{gt} ausgeführt.
Die Masken unterscheiden sich optisch stark (siehe Abb. \ref{fig:example_masks}), was sich auch in starken Unterschieden in der \ac{ipq} äußert.
\begin{figure}[htbp]
    \centering
    \begin{subfigure}[t]{0.24\textwidth}
        \includegraphics[width=\linewidth]{Figures/Deepcell_Mask_Example.png}
        \caption{Deepcell Maske}
        \label{fig:DeepcellMaske}
    \end{subfigure}
    \hfill
    \begin{subfigure}[t]{0.24\textwidth}
        \includegraphics[width=\linewidth]{Figures/nnUNet_Mask_Example.png}
        \caption{nnUNet Maske}
        \label{fig:nnUNetMaske}
    \end{subfigure}
    \hfill
    \begin{subfigure}[t]{0.24\textwidth}
        \includegraphics[width=\linewidth]{Figures/CellposeSam_Mask_Example.png}
        \caption{CellposeSam Maske}
        \label{fig:CellposeSamMaske}
    \end{subfigure}
    \hfill
    \begin{subfigure}[t]{0.24\textwidth}
        \includegraphics[width=\linewidth]{Figures/gt_Mask_Example.png}
        \caption{\ac{gt} Maske}
        \label{fig:gtMaske}
    \end{subfigure}
    \caption{Darstellung der Segmentierungsmasken der verschiedenen Segmentierungsnetze als Konturen auf einem zweidimensionalen Durchschnitt einer Stichprobe des S\_BIAD1518 Datensatz.}
    \label{fig:example_masks}
\end{figure}
\newline
Die Ergebnisse jedes Segmentierungsnetzes sind einzeln und für jedes Bild im Appendix \ref{supp:ipq} angehängt. 
Eine Zusammenfassung der Ergebnisse ist in den Boxplots in Abb. \ref{fig:boxplots_ipq} gegeben.
Das zentrale Ergebnis ist, dass CellposeSAM die besten \ac{ipq}-Werte erzielt. 
Mit einem Mittelwert von \num{0.64} ist die \ac{ipq} von CellposeSAM signifikant höher, als der Mittelwert bei nnUNet (\num{0.04}) und Deepcell (\num{0.02}), mit entsprechenden p-Werten von \num{1.80e-80} bzw. \num{1.59e-81} bei einseitigen T-Tests. 
Dennoch zeigt sich, dass CellposeSAM lediglich in der Kategorie Segmentatierungs-Qualität (SQ) sowohl den höchsten Median als auch den höchsten Mittelwert erreicht. 
Die Recognition Qualität (RQ) der nnUNet-Masken ist signifikant höher als die der CellposeSAM-Masken (p-Wert: \num{3.4764e-06}), und ebenso die Injektive Qualität (IQ) der Deepcell-Masken (p-Wert: \num{6.7763e-08}).
Obwohl nnUNet und Deepcell jeweils eine Metrik dominieren wird ihr \ac{ipq}-Wert durch die beiden schlechten Faktoren stark heruntergezogen, während CellposeSAM in jeder Metrik gut, wenn auch nicht am besten, abschneidet.
Außerdem zeigen die Boxplots viele Ausreißer in den Daten, was den Unterschieden der Bildkategorien, die der Datensatz enthält, geschuldet sein könnte.

\begin{figure}[t]
    \centering
    \includegraphics[width=\linewidth]{Figures/Segmentation_Boxplots}
    \caption{Boxplots der Ergebnisse der \ac{ipq} Berechnungen mit Faktoren \textit{$k_1$}, \textit{$k_2$} und \textit{$k_3$} jeweils gleich eins. 
    Die X-Achse unterteilt die Daten in die Kriterien Segmentatierungs-Qualität (SQ), Recognition Qualität (RQ), Injektivitäts-Qualität (IQ) und \acf{ipq}, wie in der Formel \ref{eq:ipq} beschrieben. 
    Für jede Metrik sind drei farbige Boxplots zu sehen, einer für jedes Segmentierungsnetz.
    Die Boxplots visualisieren hierbei die Verteilung der Metriken.
    Die Box repräsentiert das Interquartilsintervall (25.–75. Perzentil), wobei der Median als Linie innerhalb der Box dargestellt ist. 
    Die sogenannten Whisker reichen bis zum 1.5-fachen des Interquartilsabstands über die Box hinaus. 
    Darüber hinausgehende Punkte gelten als Ausreißer und werden einzeln dargestellt.}
    \label{fig:boxplots_ipq}
\end{figure}