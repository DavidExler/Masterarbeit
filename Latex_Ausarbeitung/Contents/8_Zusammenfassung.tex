% !TeX root = ../Thesis.tex

\chapter{Zusammenfassung}\label{ch:Conclusion}

\section{Überblick}
Die nachfolgenden Abschnitte fassen die vorliegende Arbeit zusammen und geben einen Ausblick für mögliche kommende Forschung.

\section{Zusammenfassung}
Die Arbeit befasst sich mit der automatisierten Optimierung von Deep-Learing-Methoden.
In der theoretischen Grundlage sind etablierte Deep-Learing-Methoden für die Instanzsegmentierung und Klassifikation gegeben.
Vor allem Methoden die für biologische 3D-Daten geeignet sind werden beleuchtet.
Zusätzlich sind Methoden zusammengefasst, die für die Anwendung, Auswertung und Visualisierung der vorgestellten Methoden wichtig sind.
Zu den Methoden aus der Literatur wird außerdem eine Literaturrecherche angestellt, die explizit aktuelle Forschung in dem Bereich darstellt.
Hiernach wird dargestellt, was bisher die Lücken in der Literatur sind.
Für dreidimensionale Bilddaten sind zwar sowol Methoden zur Segmentierung als auch zur Klassifikation verfügbar, aber kein umfassendes Framework, dass die Anwendung, den Vergleich und die Optimierung automatisiert.
Dementsprechend wird das Ziel der vorliegenden Arbeit definiert.
Durch die Arbeit soll der Aufwand ständig wiederkehrender Überlegungen und Vergleiche eliminiert werden, indem die Methodenauswahl und Evaluation von Deep-Learning-Methoden zur panoptischen Segmentierung von biologischen 3D-Daten automatisiert wird.\\[0.5\baselineskip]
In der Methodik der Arbeit werden hierzu einige neu entwickelte Methoden eingeführt.
Die \acf{ipq} ist eine Metrik zur Bewertung von Instanzsegmentierungsmodellen in Bezug auf ihre Eignung interpretierbare Eigenschaften zu extrahieren.
Durch die Ergebnisse der durchgeführten Experimente wird deutlich, dass die \ac{ipq} effizient die Leistung von Modellen erfasst und interpretierbare Ergebnisse bietet.
Des Weiteren werden verschiedene Encoder, Klassifikations-Köpfe, Vorverarbeitungsmethoden und Vortrainingsmethoden eingeführt.
Diese Methoden sind in der 3D-Zelldaten-Pipeline implementiert, einer automatisierten Anwendung zum Vergleich der Leistungen der Methoden.
Die Anwendung umfasst außerdem die neu eingeführte Labeling-App, eine Methode zur zeiteffizienten Annotation von dreimensionalen Bild-Datensätzen. 
Mithilfe der 3D-Zelldaten-Pipeline werden Experimente an einem Datensatz von dreidimensionalen Myotuben-Zellkultur-Aufnahmen durchgeführt.
Die Ergebnisse dieser Experimente belegen die Effizienz der 3D-Zelldaten-Pipeline.
Außerdem zeigen sie, dass die Anwendung der semi-supervised-Vortrainingsmethode mit dem neu eingeführten Pseudo-Labler zwar zu einer geringeren Durchschnittsgenauigkeit, aber auch zu einer stärkeren Generalisierung des Klassifikators führt.\\[0.5\baselineskip]
In den Ergebnissen sind des Weiteren grundlegende Erkenntnisse zu den Nuclei von Myotuben-Zellkulturen ersichtlich.
Oberflächenmerkmale von Nuclei sind nicht hilfreich für die Unterscheidung von Nucleus-Klassen.
Nur anhand ihrer Geometrie wird eine konsistente Klassenentscheidung gelernt.
Außerdem sind die Marker-Kanäle für die Klassenentscheidung genauso wichtig wie der Nucleus-Kanal.

\section{Ausblick}
Die Ergebnisse haben hezeigt, dass die Geometrie der Nuclei für die Klassifikation essenziell ist.
Dementsprechend ist ein mögliches Anliegen kommender Forschung die Erweiterung der \ac{ipq}-Metrik, um Veränderungen der Geometrie durch das Segmentierungsmodell explizit zu bestrafen.
In Anbetracht der Ergebnisse der Klassifikator-Methoden kann zukünftige Forschung weitere, kleinere \ac{cnn}-Encoder in Betracht ziehen, um die Hypothese zu prüfen, dass die Encoder zu groß sind um eine sinnvolle Repäsentation der räumlich relativ kleinen Nuclei zu finden. 
Mit neuen Datensätzen kann außerdem die Hypothese geprüft werden, dass die Oberflächenmerkmale von Nuclei egal und die Geometrie ausschlaggebend sind, indem die neu eingeführten Vorverarbeitungsmethoden weiter verglichen werden. 
Besonders interessant ist für kommende Forschung eine fortgeführte Analyse der vorgestellten Vortrainingsmethoden.
Das semi-supervised-Vortraining führt zu einer geringeren Durchschnittsgenauigkeit, aber zu einer stärkeren Generalisierungsfähigkeit des Klassifikators.
Kommende Forschung kann die Methode weiter Optimieren, um die Durchschnittsgenauigkeit möglicherweise zu erhöhen und die Vorteile der Methode so besser anwendbar zu machen.% \\[0.5\baselineskip]
