% !TeX root = ../Thesis.tex

\chapter{Zusammenfassung und Ausblick}\label{ch:Conclusion}

Die vorliegende Arbeit befasst sich mit der automatisierten Optimierung von \linebreak Deep-Learning-Methoden.
In der theoretischen Grundlage sind etablierte Deep-Learning-Methoden für die Instanzsegmentierung und Klassifikation gegeben.
Vor allem Methoden, die für biologische 3D-Daten geeignet sind, werden beleuchtet.
Zusätzlich sind Methoden zusammengefasst, die für die Anwendung, Auswertung und Visualisierung der vorgestellten Methoden wichtig sind.
Zu den Methoden aus der Literatur wird außerdem eine Literaturrecherche angestellt, die explizit die aktuelle Forschung in dem Bereich darstellt.
Hiernach werden die bisherigen Lücken in der Literatur dargestellt.
Für dreidimensionale Bilddaten sind sowohl Methoden zur Segmentierung als auch zur Klassifikation verfügbar, jedoch kein umfassendes Framework, das Anwendung, Vergleich und Optimierung automatisiert.
Dementsprechend wird das Ziel der vorliegenden Arbeit definiert.
Durch die Arbeit soll der Aufwand ständig wiederkehrender Überlegungen und Vergleiche eliminiert werden, indem die Methodenauswahl und Evaluation von Deep-Learning-Methoden zur panoptischen Segmentierung von biologischen 3D-Daten automatisiert werden.\\[0.5\baselineskip]
In der Methodik der Arbeit werden hierzu einige neu entwickelte Methoden eingeführt.
Außerdem wird ein Datensatz dreidimensionaler Bildaufnahmen von Myotubenkulturen beschrieben, an dem die eingeführten Methoden demonstriert werden.
Die \acf{ipq} ist eine Metrik zur Bewertung von Instanzsegmentierungsmodellen hinsichtlich ihrer Eignung, interpretierbare Eigenschaften zu extrahieren.
Durch die drei Faktoren der Metrik, werden Fehler in den Segmentierungsmasken bestraft die zu Änderungen der Nucleivolumina, Nucleianzahl und lokalen Dichte der Nuclei verursachen.
Durch die Ergebnisse der durchgeführten Experimente wird deutlich, dass die \ac{ipq} die Leistung von Modellen effizient erfasst und interpretierbare Ergebnisse liefert.
Des Weiteren werden verschiedene Encoder, Klassifikations-Köpfe, Vorverarbeitungsmethoden und Vortrainingsmethoden eingeführt.
Diese Methoden sind in der 3D-Zelldaten-Pipeline implementiert, einer automatisierten Anwendung zum Vergleich der Leistungen der Methoden.
Die Anwendung umfasst außerdem die neu eingeführte Labeling-App, eine Methode zur zeiteffizienten Annotation dreidimensionaler Bild-Datensätze. 
Mithilfe der 3D-Zelldaten-Pipeline werden Experimente an einem Datensatz von dreidimensionalen Myotubenkultur-Aufnahmen durchgeführt.
Die Ergebnisse dieser Experimente belegen die Effizienz der 3D-Zelldaten-Pipeline.
Außerdem zeigen sie, dass die Anwendung der semi-supervised-Vortrainingsmethode mit dem neu eingeführten Pseudo-Labler zwar zu einer geringeren Durchschnittsgenauigkeit, aber auch zu einer stärkeren Generalisierung des Klassifikators führt.\\[0.5\baselineskip]
In den Ergebnissen sind des Weiteren grundlegende Erkenntnisse zu den Nuclei von \linebreak Myotubenkulturen ersichtlich.
Oberflächenmerkmale von Nuclei sind nicht hilfreich zur Unterscheidung von Nucleus-Klassen.
Nur aus ihrer Geometrie wird eine konsistente Klassenentscheidung gelernt.
Außerdem sind die Marker-Kanäle für die Klassenentscheidung genauso wichtig wie der Nucleus-Kanal.\\[0.5\baselineskip]
%Ausblick
Die Ergebnisse haben gezeigt, dass die Geometrie der Nuclei für die Klassifikation essenziell ist.
Dementsprechend ist ein mögliches Anliegen kommender Forschung die Erweiterung der \ac{ipq}-Metrik, um Veränderungen der Geometrie durch das Segmentierungsmodell explizit zu bestrafen.
Für diese Erweiterung kann ein neuer Faktor eingeführt werden, der die geometrischen Eigenschaften der Annotationen und der segmentierten Instanzen erfasst, beispielsweise als Parameter der Fourier-Entwicklung ihrer Konturen, und mit einem geeigneten Ähnlichkeitsmaß vergleicht.
In Anbetracht der Ergebnisse der Klassifikator-Methoden kann zukünftige Forschung weitere, kleinere \ac{cnn}-Encoder in Betracht ziehen, um die Hypothese zu prüfen, dass die Encoder zu groß sind um eine sinnvolle Repäsentation der räumlich relativ kleinen Nuclei zu finden. 
Mit neuen Datensätzen kann außerdem die Hypothese geprüft werden, dass die Oberflächenmerkmale von Nuclei für die Klassifikation unwichtig und die Geometrie ausschlaggebend sind, indem die neu eingeführten Vorverarbeitungsmethoden weiter verglichen werden. 
Besonders interessant ist für die kommende Forschung eine fortgeführte Analyse der vorgestellten Vortrainingsmethoden.
Das semi-supervised-Vortraining führt zu einer geringeren Durchschnittsgenauigkeit, aber zu einer stärkeren Generalisierungsfähigkeit des Klassifikators.
Kommende Forschung kann die Methode weiter optimieren, um die Durchschnittsgenauigkeit möglicherweise zu erhöhen und die Vorteile der Methode so besser anwendbar zu machen.
Hierzu können weitere Eigenschaften der Nuclei erfasst und dem Merkmalsvektor hinzugefügt werden, auf den der Label-Spreading-Algorithmus angewandt wird, oder die PCA durch einen anderen Algorithmus ersetzt werden.
Außerdem kann die anschließende Trainingsroutine mit mehr Epochen und geringerer Lernrate durchgeführt werden.\\[0.5\baselineskip]
In Ausblick auf eine vollständige Instanzsegmentierung von Myotuben kann die panoptische Segmentierungsmaske der Nuclei verwendet werden, um Hinweise für ein Segmentierungsmodell wie \ac{sam} zu generieren.
Die Myotuben-Zellkerne liegen innerhalb der Myotuben und sind mit ihrer Längsachse entlang der Hauptausrichtung der Myotuben orientiert.
Anhand der Ausrichtungen und relativen Positionen der vorhergesagten Myotuben-Zellkerne lässt sich eine polynomische Kurve bestimmen, deren Verlauf weitgehend glatt ist.
Diese Kurven können dem Segmentierungsmodell als Hinweise übergeben und so eventuell die Instanzsegmentierung von Myotuben ermöglichen.
In einigen ersten Versuchen hiervon konnten zwar einige intuitiv sinnvolle Polynome extrahiert werden, aber auch einige fehlerhafte.
Außerdem werden die Myotuben selbst mit den Polynomen als Hinweis durch das \ac{sam}-Modell nicht perfekt instanzsegmentiert.