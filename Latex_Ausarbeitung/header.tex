% !TeX root = thesis.tex


\documentclass[12pt, table, twoside, headsepline, openright, english, fleqn,numbers=noenddot]{scrreprt}
\usepackage{
    acronym,
	amsmath,
	amssymb,
	blindtext,
	booktabs,
	caption,
	csquotes,
	fontspec,
	geometry,
	graphicx,
	hyperref,
	mathtools,
	microtype,
	multirow,
	paralist,
	tikz,
	pgfplots,
	scrlayer-scrpage,
	setspace,
	siunitx,
	subfig,
	tabu,
	threeparttablex,
	unicode-math,
	url,
	wallpaper,
	xparse
}
\usepackage{algorithm}
\usepackage{algorithmic}

\usepackage[main=ngerman, english]{babel}
\usepackage[xspace]{ellipsis}
\usepackage[nottoc, notlot, notlof, chapter]{tocbibind}

% Fonts
\usepackage{libertine}
\setmathfont[math-style=ISO,bold-style=ISO]{ XITS Math }

\definecolor{slategray}{rgb}{0.41, 0.41, 0.41}

\RedeclareSectionCommand[ afterskip=.7\baselineskip]{chapter}

% New chapter format
\renewcommand*{\chapterformat}{\chapternumber{\thechapter}}
\renewcommand{\chapterlinesformat}[3]{\chaptertitle{#3}#2\\
  \rule[.5\baselineskip]{16.4cm}{1pt}\\*
  }

% Chapter number
\newcommand{\chapternumber}[1]{%
    \usekomafont{chapter}%
    \mdseries%
    \begin{minipage}[b]{0.15\textwidth}%
        \raggedright{%
            \hspace{7mm}%
            %{\color{gray}\fontsize{60}{60}\selectfont#1}%
            {\color{slategray}\fontsize{50}{50}\selectfont#1}%
        }%
    \end{minipage}%
}

% Chapter title
\newcommand{\chaptertitle}[1]{%
    \usekomafont{chapter}%
    \leavevmode\smash{\begin{minipage}[b]{0.95\textwidth}%
        \raggedleft #1%
    \end{minipage}}%
}


% pgfplots
\pgfplotsset{compat=1.14} % or newer, just test it
\pgfplotsset{every x tick label/.append style={font=\footnotesize, yshift=0.1ex}}
\pgfplotsset{every y tick label/.append style={font=\footnotesize, xshift=0.1ex}}
\pgfplotsset{every z tick label/.append style={font=\footnotesize, xshift=0.1ex}}
\pgfplotsset{every axis label/.append style={font=\footnotesize}}
\pgfplotsset{every axis legend/.append style={font=\scriptsize}}
\pgfplotsset{every axis/.append style={tick style={thin,black}}}
\newlength\figureheight
\newlength\figurewidth
%\usepgfplotslibrary{external}
%\tikzexternalize[prefix=External/]


% References
%\usepackage[backend=biber, style=ieee]{biblatex}
\usepackage[backend=biber, style=numeric]{biblatex}
%\usepackage[backend=biber, style=apa]{biblatex}
%\DeclareLanguageMapping{ngerman}{english-apa}
%\DeclareLanguageMapping{english}{english-apa}
\bibliography{mybib}
\DefineBibliographyStrings{ngerman}{
   andothers = {{et\,al\adddot}}, % et al statt u.a.
} 

% Type area
\areaset[1.3cm]{15.3cm}{23.5cm}
\KOMAoptions{toc=bibliographynumbered, toc=listof, headwidth=16.7cm:0cm, footwidth=16.7cm:0cm}


% Head/foot
\setkomafont{pagehead}{\normalsize \normalfont}
\setkomafont{pagefoot}{\normalsize}
\pagestyle{scrheadings}
\chead[]{}
\lehead[]{\quad \headmark}
\rohead[]{\headmark \quad}
\ihead[]{}
\cfoot[]{}
\lefoot[]{\quad \pagemark}
\rofoot[]{\pagemark \quad}
\automark[section]{chapter}
\setlength{\footheight}{28.99998pt}


% Captions
\DeclareCaptionFormat{myFormat}{#1#2 \textbf{|} #3\par}
\DeclareCaptionFormat{myFormat2}{#1#2 #3\par}
\captionsetup{labelfont={bf,footnotesize},textfont={footnotesize},format=myFormat,labelsep=none}
\addto\captionsngerman{
	\renewcommand{\figurename}{Abb.}
	\renewcommand{\tablename}{Tab.}
}
\captionsetup[subfigure]{textfont={scriptsize},labelfont={scriptsize},format=myFormat2,labelsep=none,labelformat=brace}
%\captionsetup[subfloat]{labelfont={bf,footnotesize},textfont={footnotesize},labelformat=simple,format=plain}


% Links and pdf description
\hypersetup{
	pdfencoding=auto,
	linkcolor=blue!50!black!100,
	colorlinks=true,
	citecolor=blue!50!black!100,
	urlcolor=blue!50!black!100,
	pdftitle={3D-Segmentierung und Merkmalsextraktion in Mikroskopiedaten},
	pdfauthor={David Exler},
	pdfsubject={},
	pdfkeywords={Master's thesis, Segmentierung, 3D-Segmentierung, Zellmikroskopie, Cellpose, Stardist},
}


% Autorefnames
\addto\extrasngerman{% \extrasenglish for German language
	\renewcommand{\equationautorefname}{Eq.}%
	\renewcommand{\chapterautorefname}{Chapter}%
}


% siunitx
\sisetup{
	locale=DE, % locale=DE for German language
	separate-uncertainty,
	exponent-product=\cdot
}


% Definitions
\catcode`√=\active
\let√\sqrt
\catcode`•=\active
\let•\item
\catcode`…=\active
\let…\dots
\newcommand{\vect}[1]{\symbf{#1}}
\DeclareMathOperator*{\argmin}{arg\,min}

\newcommand{\CC}{C\nolinebreak\hspace{-.04em}\raisebox{.1ex}{\small +}\nolinebreak\hspace{-.08em}\raisebox{.1ex}{\small +}}

\pltopsep=5pt
\plitemsep=5pt
